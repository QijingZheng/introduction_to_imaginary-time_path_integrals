\begin{frame}
  \frametitle{Multiparticle Generalization}
  For a system of $N$ \emph{distinguishable} nuclei,
  \footnote{
    We are neglecting the exchange effect, which is generally fine for nuclei
    (unless treating \SI{4}{\kelvin} Helium) but not for electrons.
    \medskip
% The effect of including exchange is to allow ring
% polymers representing different particle to “bond” with each other (see Ethan’s talk).
  }
  \begin{equation*}
    \hamil_P = \sum_I^N \frac{\mathbf{p}_I^2}{2m_I}
    +
    \underline{
      \hat{V}\tikzmark{n_particle_V}(\mathbf{x}_1,\ldots,\mathbf{x}_N)
    },
  \end{equation*}

  \begin{tikzpicture}
    [overlay, remember picture]
    \node[
    draw=black, rounded corners=3pt, below right = 1em and 10em of
    n_particle_V,
    % text width=4.25cm,
    align=left, % inner sep=0pt
    anchor=north,
    ] (n_particle_note1) {
      \begin{minipage}{0.40\linewidth}
        \tiny
        \textbf{Potential energy:} could be from emperical potential or from \abinitio{}
        calculations. \emph{Born-Oppenheimer approximation} implicitly implied.
      \end{minipage}
    };                     
    \draw[red,->, >=stealth, in=-90, out=180] (n_particle_note1.west) to ($
    (n_particle_V.north east) + (3pt, -6pt)
    $);
  \end{tikzpicture}

  The PIMD Hamiltonian is
  \begin{equation*}
    H_P(\{\mathbf{x}\}, \{\mathbf{p}\}) =
    \sum_{\tikzmark{particle_ind}I=1}^N \sum_{\tikzmark{bead_ind}i=1}^P
    \left[
      \frac{(\mathbf{p}^i_I)^2}{2\tilde{m}_I^i}
      +
      \frac{1}{2} m_I \tikzmark{omegaP}\omega_P^2 (\mathbf{x}^{i+1}_I - \mathbf{x}^i_I)^2
    \right]
      +
      \sum_{i=1}^P 
      \hat{V}(\mathbf{x}_1^i,\ldots,\mathbf{x}_N^i)
  \end{equation*}

  \begin{tikzpicture}
    [overlay, remember picture]

    \node[
    draw=black, rounded corners=3pt, below left = 2.5em and 4em of
    particle_ind,  align=center, % inner sep=0pt
    anchor=south,
    ] (index_particle_note) {
      \small sum over nuclei index
    };                     

    \node[
    draw=black, rounded corners=3pt, below right = 2.5em and 4em of
    bead_ind,  align=center, % inner sep=0pt
    anchor=south,
    ] (index_bead_note) {
      \small sum over bead index
    };                     

    \node[
    draw=black, rounded corners=3pt, right = 2.5em of index_bead_note,
    align=center, % inner sep=0pt
    % anchor=south,
    ] (omegaP_note) {
      \small $\omega_P = P k_B T / \hbar$
    };                     

    \draw[red,->, >=stealth, in=-90, out=90] (index_particle_note.north) to ($
    (particle_ind.south) + (6pt, -3pt)
    $);
    \draw[red,->, >=stealth, in=-90, out=90] (index_bead_note.north) to ($
    (bead_ind.south) + (6pt, -3pt)
    $);
    \draw[red,->, >=stealth, in=-90, out=90] (omegaP_note.north) to ($
    (omegaP.south) + (6pt, -3pt)
    $);
  \end{tikzpicture}
  
  \vspace{12pt}
  \begin{figure}
    \centering
    \resizebox{0.85\textwidth}{!}{
    \begin{tikzpicture}
        [
        spring/.style={
        line width=0.5pt, decorate,
        decoration={
            coil, amplitude=3.0, 
            segment length=2.5
        }}, 
        potential/.style={
        line width=0.5pt, decorate,
        color=green!50!black, dashed,
        decoration={
            snake, amplitude=2.0, 
            segment length=100
        }}
        ]
        \def\pointsA{
            1.0000/   0.0000,
            0.5136/   0.8896,
            -0.5112/   0.8853,
            -1.0629/   0.0000,
            -0.6401/  -1.1086,
            0.5326/  -0.9226
        }
        \def\rpoffset{(4.0, -0.5)}

        % Draw the potential
        % \draw [line width=2pt, domain=-12.5:-7.5] plot [samples=300] (\x, {0.5 * (\x + 10) * (\x + 10) - 1.5});
        % \draw [line width=2pt, domain=-2.5:2.5] plot [samples=300] (\x, {0.5 * (\x + 0) * (\x + 0) - 1.5});
        % Draw an arrow
        \draw [line width=1.5pt, ->, >=stealth] (-5, 0) -- (-2, 0);

        % First, draw the springs for the first RP
        \draw[spring, color=red] (  1.0000,   0.0000) 
                            -- (  0.5136,   0.8896) 
                            -- ( -0.5112,   0.8853) 
                            -- ( -1.0629,   0.0000) 
                            -- ( -0.6401,  -1.1086) 
                            -- (  0.5326,  -0.9226) 
                            -- cycle;
        % First, draw the springs for the second RP
        \draw[spring, color=blue] ($ (  1.0000,   0.0000) + (4.0, -0.5) $)
                              -- ($ (  0.5136,   0.8896) + (4.0, -0.5) $)
                              -- ($ ( -0.5112,   0.8853) + (4.0, -0.5) $)
                              -- ($ ( -1.0629,   0.0000) + (4.0, -0.5) $)
                              -- ($ ( -0.6401,  -1.1086) + (4.0, -0.5) $)
                              -- ($ (  0.5326,  -0.9226) + (4.0, -0.5) $)
                              -- cycle;

        % draw the interaction between the two atoms
        \draw[potential] (-10, 0) -- node[above, sloped]
        {$\hat{V}(\mathbf{x}_1,\ldots,\mathbf{x}_N$)} ++\rpoffset;
        \draw[potential, line width=0pt] (-0.64, -1.11) -- node[below, sloped]
        {$\hat{V}(\mathbf{x}_1^i,\ldots,\mathbf{x}_N^i$)} ++\rpoffset;
        % Draw the single particles
        \shade[shading=ball, ball color=blue!50!cyan] (-10, 0) circle (8pt);
        \shade[shading=ball, ball color=red] ($ (-10, 0.0) + (4.0, -0.5) $) circle (8pt);

        \foreach \x/\y in \pointsA {
            \coordinate (A) at (\x, \y);
            % draw the interaction between the two atoms
            \draw[potential] (A) -- ++\rpoffset;
            % Ring-Polymer of the first atom
            \shade[shading=ball, ball color=blue!50!cyan] (A) circle (8pt);
            % Ring-Polymer of the second atom
            \shade[shading=ball, ball color=red] ($ (A) + (4.0, -0.5) $) circle (8pt);
        }
        \node[] at (-8, -2.5) {
        % \fontsize{24}{12pt}\selectfont
        \huge
        % $Z = \tr \left[e^{-\beta\hamil}\right]$
        quantum particle
        };
        \node[text width=5cm, align=center] at (2.0, -2.5) {
        % \fontsize{24}{12pt}\selectfont
        \huge
        % configuration integral of classical ring-polymer
        classical ring-polymers
        };
      \end{tikzpicture}
    }
  \end{figure}
\end{frame}
