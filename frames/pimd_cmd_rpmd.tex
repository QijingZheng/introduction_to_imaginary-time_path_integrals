\begin{frame}
  \frametitle{PIMD, (PA)CMD and RPMD --- Fictitious Masses}
  The RP Hamiltonian in the normal-mode coordinates
  \begin{align*}
    H_P(\{\mathbf{\tilde{x}}\}, \{\mathbf{\tilde{p}}\}) &=
    \sum_{i=1}^P
    \left[
      \frac{\tilde{p}_i^2}{2\tilde{m}_i}
      +
      \frac{1}{2} m \Omega_i^2 \tilde{x}_i^2
    \right]
      +
      \sum_{i=1}^P V\left(x_i(\tilde{x}_1,\ldots,\tilde{x}_P)\right) \cr
      \Rightarrow \quad
      \tilde{m}_i \ddot{\tilde{x}}_i &=
                           -m
                           \tikzmark{RPOmega}\Omega_i^2
                           \tilde{x}_i
                           - \sum_{j=1}^P
                           {
                           \partial
                           V\left(x_j(\tilde{x}_1,\ldots,\tilde{x}_P)\right)
                           \over
                           \partial
                           \tilde{x_i} \tikzmark{RPPot}
                           } \cr
      \Rightarrow \quad
      \tilde{m}_1 \ddot{\tilde{x}}_1 &=
      - {1\over\sqrt{P}} \sum_{j=1}^P {\partial V(x_j) \over \partial x_j}
  \end{align*}

  \begin{tikzpicture}
    [overlay, remember picture]

    % \node[
    % draw=black, rounded corners=3pt, below left = 1em and 2em of
    % RPOmega,
    % align=left,
    % anchor=north,
    % ] (rp_omega_note_1) {
    %   $\Omega_1 = 0$
    % };                     
    % \draw[red,->, >=stealth, in=-90, out=0] (rp_omega_note_1.east) to ($
    % (RPOmega) + (3pt, -6pt)
    % $);
    
    \node[
    draw=black, rounded corners=3pt, right = 5em of RPPot,
    align=left,
    % anchor=north,
    ] (rp_pot_note_1) {
      $
      =\sum_{j=1}^P {\partial V(x_j) \over \partial x_j} U^T_{ji}
      $
    };                     
    \draw[red,->, >=stealth, in=-90, out=180] (rp_pot_note_1.west) to ($
    (RPPot) + (-3pt, -4pt)
    $);
  \end{tikzpicture}
  % Note that $\tilde{x}_1 = 1/\sqrt{U} \sum_{j=1}^P x_j$ and $\Omega_1 = 0$, we have
  % \begin{equation*}
  %     \tilde{m}_1 \ddot{\tilde{x}}_1 =
  %     - {1\over\sqrt{P}} \sum_{j=1}^P {\partial V(x_j) \over \partial x_j}
  % \end{equation*}

  \begin{itemize}
  \item \textcolor{blue}{{PIMD}} --- scale the mass to bring all the frequencies
    to the same value $\omega_p$.
    \begin{equation*}
      \tilde{m}_1 = m, \qquad \tilde{m}_i = 4\sin^2\left({(i-1)\pi\over
          P}\right) m \quad (2 \le i \le P)
    \end{equation*}
  \item \textcolor{blue}{{(PA)CMD}} --- adiabatic parameter $\gamma > 1$,
    shifting the noncentroid frequencies $\gamma\omega_P$ off the physical
    spectra $\omega_{\text{max}}$.
    \begin{equation*}
      \tilde{m}_1 = m, \qquad \tilde{m}_i = 4\sin^2\left({(i-1)\pi\over
          P}\right) m / \gamma^2 \quad (2 \le i \le P)
    \end{equation*}
  \item \textcolor{blue}{{RPMD}} --- Use real masses for all the beads.
    \begin{equation*}
      \tilde{m}_i = m, \quad (1 \le i \le P)
    \end{equation*}
  \end{itemize}
\end{frame}
%%% Local Variables:
%%% mode: latex
%%% TeX-master: "auto/pimd_cmd_rpmd"
%%% End:
