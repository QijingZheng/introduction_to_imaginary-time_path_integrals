\begin{frame}
  \frametitle{PIMD --- Introducing Momenta}
  % Note the standard Gaussian integral
  % \begin{equation*}
  %   \label{eq:gauss_int}
  %   \left({\beta\over2\pi {\tilde{m}}}\right)^{1/2}\int\dt{p}\,e^{-\beta p^2/2\tilde{m}} = 1
  % \end{equation*}
  Insert $P$ Gaussian integral into $Z$:
  $\displaystyle\quad
  \left({\beta\over2\pi {\tilde{m}}}\right)^{1/2}\int\dt{p}\,e^{-\beta {p^2\over2\tilde{m}}} = 1
  $
  \begin{align*}
    Z &=
    % \left({1\over2\pi\hbar}\sqrt{m\over\tikzmark{mass_matrix}\tilde{m}}\right)^P
        \left({1\over2\pi\hbar}\right)^P\prod_{i=1}^P\sqrt{m\over\tikzmark{mass_matrix}\tilde{m}_i}
         \lim_{P\to\infty} 
         \int\dt{x_1}\ldots\int\dt{x_P}
         \int\dt{p_1}\ldots\int\dt{p_P}\,\cr
    & \phantom{
      % \left({1\over2\pi\hbar}\sqrt{m\over\tilde{m}}\right)^P
      \left({1\over2\pi\hbar}\right)^P\prod_{i=1}^P\sqrt{m\over\tilde{m}_i}
      \lim_{P\to\infty}}
      \exp\left\{
          \displaystyle
          -\beta_P
          \sum_{i=1}^P \left[
            {p_i^2\over 2\tikzremember{pseudo_mass}{$\tilde{m}_i$}} +
            {\tikzremember{real_mass}{$m$}\over2(\beta_P\hbar)^2} (x_{i+1} - x_i)^2 +
            V(x_i)
          \right]
        \right\}\cr
    &= {\cal C} \lim_{P\to\infty}
      \int\dt{\mathbf{x}^P}\int\dt{\mathbf{p}^P}
      \exp\left[-\beta_P H_P(\mathbf{x}, \mathbf{p})\right]
  \end{align*}

  \begin{tikzpicture}[overlay, remember picture]
    \node[
    draw=black, rounded corners=3pt, above right = 1em and 5em of
    real_mass, align=center, % inner sep=0pt
    ] (note1) {
      \small real physical mass
    };                     
    \draw[red,->, >=stealth, in=90, out=-90] (note1.south west) to ($
    (real_mass.north east) + (-3pt, 3pt)$ );
    % \draw[red,->, >=stealth] (note1.south) -- ++(0, -2.0em) -| (real_mass.north east);

    \node[
    draw=black, rounded corners=3pt, below right = 0.8em and 2em of
    pseudo_mass, text width=4.5cm, align=center,
    ] (note2) {
      \begin{minipage}{1.0\linewidth}
        \small $\tilde{m}_i$ need \emph{not} be physical. More general form:
        \(\displaystyle
          \frac{1}{2}\mathbf{p}^T \mathrm{M}^{-1} \mathbf{p}
        \), where $\mathrm{M}$ is the mass matrix, and $\mathbf{p} = (p_1,\ldots,p_P)$.
      \end{minipage}
    };                     
    \draw[red,->, >=stealth, in=-90, out=180] (note2.west) to ($
    (pseudo_mass.south east) + (-3pt, -3pt) $);

    \node[
    draw=black, rounded corners=3pt, below left = 1.0em and 1em of
    mass_matrix, align=center,
    ] (note3) {
        \small More general: \\[3pt]
        $\prod_i{\tilde{m}_i} \Rightarrow \det[\mathrm{M}]$
    };                     
    \draw[red,->, >=stealth, in=-90, out=0] (note3.east) to ($
    (mass_matrix.south) + (3pt, -3pt) $);
    
    % \node (kaka) {kaka};
    % \draw[blue] ($ (kaka.south west) + (0.0, -3pt) $) -- ($ (kaka.south east) + (0.0, -3pt) $);
    
  \end{tikzpicture}

  \vspace{6pt}
  \begin{itemize}
    \item PIMD trajectories are obtained by integrating Hamilton's classical
      equaiton of motion
      % \footnote{
      % $\beta_P H_P(\mathbf{x}, \mathbf{p})$ can also be defined as
      % $\beta\left( H_P(\mathbf{x}, \mathbf{p})/P\right)$, which affects the
      % dynamics but not the statistics.
      % }
      \begin{equation*}
        {\dt{p_i}\over\dt{t}} = {}+{\partial H_P(\mathbf{x},
          \mathbf{p})\over\partial x_i};
        \qquad
        {\dt{x_i}\over\dt{t}} = {}-{\partial H_P(\mathbf{x},
          \mathbf{p})\over\partial p_i}
      \end{equation*}

    \item $\beta_P H_P(\mathbf{x}, \mathbf{p}) \Rightarrow
      \beta\left( H_P(\mathbf{x}, \mathbf{p})/P\right)$, which affects the
      dynamics but not the statistics.
    \item Different mass matrices give the same static average, only
      dynamics will be changed.
    \item Mathematically, the main difference between PIMD, PACMD and RPMD is the
      choice of the mass matrix. Physically, they differ dramatically!
  \end{itemize}
  
\end{frame}
