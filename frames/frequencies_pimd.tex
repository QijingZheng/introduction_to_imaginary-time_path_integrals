\begin{frame}
  \frametitle{Frequencies in PIMD}

  For the 1D harmonic oscillator: $V(x) = \frac{1}{2}m\omega x^2$, 
  the normal mode frequencies
  \begin{equation*}
    \tilde{\Omega}_i = \sqrt{\Omega_i^2 + \omega^2};\qquad \Omega_i = 2\omega_P \sin\left({(i-1)\pi\over P}\right)
  \end{equation*}
  Hence the highest frequency is
  \begin{equation*}
    \tilde{\Omega}_{\text{max}} = \sqrt{\Omega_{\text{max}}^2 + \omega^2} =
    \sqrt{4\omega_P^2 + \omega^2}
  \end{equation*}
  If we use the typical convergence criteria: $P = \alpha\hbar\omega / k_BT$,
  where $\alpha > 1$ determines how accurate the calculation is.
  \begin{equation*}
    \tilde{\Omega}_{\text{max}} = (4\alpha^2 + 1)^{1\over2}\omega
  \end{equation*}

  \centerline{
    \tikz \draw[black, line width=1pt, baseline] (0.0, 0.0) -- node[pos=0.5,
    align=center, above] {\textbf{}} (8.0, 0.0);
  }
  
  \begin{multicols}{2}
    \begin{itemize}
    \item For $\alpha = 1$, the highest frequency in PIMD is $\sqrt{5}$ times larger
      than classical MD.
    \item For a \emph{naive} implementation of PIMD, the time step should be $\sqrt{5}$ times
      smaller than classical MD.
    \vfill\null
    \columnbreak
    \item Methods to allow bigger time-steps:
      \begin{enumerate}
      \item  Scale the normal mode masses so they all oscillate at the same
        frequency.
      \item Multiple time-scale molecular dynamics: use smaller time-steps for
        bead forces.
      \item Use an integrator where the free ring polymer is evolved exactly.
      \end{enumerate}

    \end{itemize}
    
  \end{multicols}
\end{frame}
%%% Local Variables:
%%% mode: latex
%%% TeX-master: t
%%% End:
