\begin{frame}
  \frametitle{Terminologies and Properties of the Ring-Polymer}

  \begin{figure}
    \centering
    \begin{tikzpicture}
      [
      spring/.style={
        line width=0.5pt, decorate,
        decoration={
          coil, amplitude=3.0, 
          segment length=2.5
        }}, 
      potential/.style={
        line width=0.5pt, decorate,
        color=green!50!black,
        decoration={
          snake, amplitude=2.0, 
          segment length=100
      }}
      ]
      \def\pointsA{
            1.0000/   0.0000,
            0.5136/   0.8896,
           -0.5112/   0.8853,
           -1.0629/   0.0000,
           -0.6401/  -1.1086,
            0.5326/  -0.9226
      }
      % First, draw the springs for the first RP
      \draw[spring, color=red] (  1.0000,   0.0000) 
                            -- (  0.5136,   0.8896) 
                            -- ( -0.5112,   0.8853) 
                            -- ( -1.0629,   0.0000) 
                            -- ( -0.6401,  -1.1086) 
                            -- (  0.5326,  -0.9226) 
                            -- cycle;
      \foreach \x/\y in \pointsA {
        \coordinate (A) at (\x, \y);
        \shade[shading=ball, ball color=blue!50!cyan] (A) circle (8pt);
      }

      \draw[black, line width=1pt, dashed] (0.0, 0.0) circle (1.1);
      \draw[black, line width=1pt] (0.0, 0.0) circle (0.05);
      \draw[black, line width=1pt, ->, >=stealth] (30:0.05) -- node[above,
      sloped, pos=0.3] {$r_G$} (30:1.1);

      % \node[] at (0.1, -0.4) {$r_G$};

      \node[] at (90:2.5) {
        \begin{minipage}{0.95\textwidth}
          $\bullet\kern1ex$\textbf{Radius of Gyration} -- the spread in imaginary
          time. For a free particle the root mean square radius of gyration is:
          \begin{equation*}
            \braket{r_G^2(T)}^{1/2} = {\Lambda(T)\over\sqrt{8\pi}}
            \qquad \qquad
            \Lambda(T) = {h\over\sqrt{2\pi m k_B T}}
          \end{equation*}
        \end{minipage}
      };

      \node[align=left] at (0:4.0) {
        \begin{minipage}{0.3\textwidth}
          $\bullet\kern1ex$\textbf{Centroid} -- the center of the Ring-Polymer.
          \begin{equation*}
            x_c = {1\over P}\sum_{i=1}^P x_i
          \end{equation*}
        \end{minipage}
      };

      \node[align=left] at (180:4.0) {
        \begin{minipage}{0.3\textwidth}
          $\bullet\kern1ex$\textbf{Bead to bead distance} -- For a free particle
          the average is:
          \begin{equation*}
            \sqrt{\beta\hbar^2\over P m}
          \end{equation*}
          Note the distance decreases as $P$ increases.
        \end{minipage}
      };

      \node[align=center] at (270:2.5) {
        \begin{minipage}{0.8\textwidth}
          \begin{center}
            $\bullet\kern1ex$\textbf{Bead spring constants}\thinspace\footnotemark\thinspace -- determined by
            $m\omega_P^2$
            \begin{equation*}
              \omega_P = {1\over\beta_P\hbar} = Pk_B T / \hbar
            \end{equation*}
          \end{center}
        \end{minipage}
      };
    \end{tikzpicture}
  \end{figure}
  \footnotetext{
    In some textbooks, $\omega_P = \sqrt{P} k_B T / \hbar$ if the Hamiltonian is
    $H_P(\mathbf{x}, \mathbf{p})/P$
  }
  % \vspace{12pt}
  \begin{center}
    \fbox{
      The overall object is referred to as an \emph{Imaginary Time Path}
      or a \emph{Ring-Polymer}.
    }
  \end{center}
\end{frame}
