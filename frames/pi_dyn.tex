\begin{frame}
  \frametitle{PIMD, CMD and RPMD
  \footnote{
    \begin{minipage}[t]{1.0\linewidth}
    \textit{J. Chem. Phys.}, \textbf{130}, 194510(2009); \\
    \textit{J. Chem. Phys.}, \textbf{129}, 074501(2008); \\
    \end{minipage}
    \smallskip
  }
  }

  % \begin{center}
  %   \fbox{
  %       Standard PIMD gives exact \textcolor{blue}{static equilibrium}
  %       properties not the \textcolor{red}{dynamical} ones!
  %   }
  % \end{center}

  \begin{center}
    
    \resizebox{0.94\textwidth}{!}{
    \begin{tikzpicture}
      [
      every annotation/.style = {font = \small},
      spring/.style={
        line width=0.5pt,
        decorate,
        decoration={
            coil,
            amplitude=3.5, 
            segment length=5.0,
          }
        }, 
      ]
        \fill[black!20!white] (0, 0) circle(1.2);
        \fill[white] (0, 0) circle(0.6);

        \draw[spring, draw=red] (60:1.35) arc[
            radius=1.35, start angle=60, end angle=180
        ];
        \draw[spring, draw=green] (180:1.35) arc[
            radius=1.35, start angle=180, end angle=300
        ];
        \draw[spring, draw=blue] (300:1.35) arc[
            radius=1.35, start angle=-60, end angle=60
        ];

        \foreach \a/\c in {60/red!40!white, 180/green!40!white, 300/blue!40!white}
        {
            \fill[white] (\a:1.35) circle (.5);
            \fill[\c] (\a:1.35) circle (.4);
        }
        \coordinate (pimd) at (60:1.35);
        \coordinate (cmd)  at (180:1.35);
        \coordinate (rpmd) at (300:1.35);
        
        \node[align=center] (NQEs) at (0.0, 0.0) {\textcolor{blue}{NQEs}};

        \node[
            inner sep=3pt,
            draw, line width=0.5pt,
            rounded corners=3pt,
            anchor=south west,
            annotation,
            text width=13em
        ] at ($ (pimd) + (60:0.15) $) {
            \centerline{\normalsize \textcolor{blue}{PIMD}}
            \vspace{-\medskipamount}
            \begin{list}{\zjball{red}}{\
                \topsep=0pt\itemsep=3pt\parsep=0pt
                \parskip=0pt\labelwidth=8pt\leftmargin=8pt
                \itemindent=0pt\labelsep=2pt
            }
            \item Static \textcolor{blue}{equilibrium} properties.
            \item Need to efficiently and ergodically sample the phase surface.
              \smallskip
                \begin{list}{\zjball{blue}}{\
                \topsep=0pt\itemsep=3pt\parsep=0pt
                \parskip=0pt\labelwidth=8pt\leftmargin=8pt
                \itemindent=0pt\labelsep=2pt
                }
                \item Thermostatting
                \item Coordinate transformation and bead masses scaling
                \end{list}
            \end{list}
        };

        \node[
            inner sep=3pt,
            draw, line width=0.5pt,
            rounded corners=3pt,
            annotation, anchor=north west,
            text width=13em
        ] at ($ (rpmd) + (300:0.15) $) {
          \centerline{\normalsize \textcolor{blue}{RPMD}
            % \footnote[frame]{
            %   \textit{J. Chem. Phys.}, \textbf{121}, 3368 (2004);\enspace
            %   \textit{J. Chem. Phys.}, \textbf{125}, 124105 (2006)
            % }
          }
            \vspace{-\medskipamount}
            \begin{list}{\zjball{red}}{\
                \topsep=0pt\itemsep=3pt\parsep=0pt
                \parskip=0pt\labelwidth=8pt\leftmargin=8pt
                \itemindent=0pt\labelsep=2pt
            }
                \item Approximate quantum \textcolor{red}{dynamics}
                \item PIMD with \emph{physical} bead masses.
                \item Thermostats should \emph{not} be used.
                \item Must sample initial conditions.
            \end{list}
        };

        \node[
            inner sep=3pt,
            draw, line width=0.5pt,
            rounded corners=3pt,
            annotation, anchor=east,
            text width=16em
        ] at ($ (cmd) + (180:0.30) $) {
          \centerline{\normalsize \textcolor{blue}{CMD}
            % \footnote{
            %   J. Chem. Phys. 100, 5093 (1994)
            % }
          }
            \vspace{-\medskipamount}
            \begin{list}{\zjball{red}}{\
                \topsep=0pt\itemsep=3pt\parsep=0pt
                \parskip=0pt\labelwidth=8pt\leftmargin=8pt
                \itemindent=0pt\labelsep=2pt
            }
            \item Approximate quantum \textcolor{red}{dynamics}
            \item Path centroid idea
            \begin{equation*}
                m \ddot{x}_c = \left\langle
                \delta(x_0 - x_c) {-1\over P}
                \sum_{i=1}^P {\partial V(x_i) \over \partial x_i}
                \right\rangle
            \end{equation*}
            
            \item Sample the whole configuration space available to the non-centroid modes
              at each given position of centroid.

            \end{list}

            \smallskip
            \centerline{\textcolor{blue}{(PA)CMD} }
            \begin{list}{\zjball{red}}{\
                \topsep=0pt\itemsep=3pt\parsep=0pt
                \parskip=0pt\labelwidth=8pt\leftmargin=8pt
                \itemindent=0pt\labelsep=2pt
            }
              \item PIMD with \emph{adiabatic decoupling} of centroid and
                non-centroid motions.
              \item \emph{No} thermostat for the centroid mode.
              \item Thermostatting for noncentroid mode as in PIMD.
              \item Must sample initial conditions.
           \end{list}
        };
    \end{tikzpicture}
    }
  \end{center}

\end{frame}
%%% Local Variables:
%%% mode: latex
%%% TeX-master: t
%%% End:
