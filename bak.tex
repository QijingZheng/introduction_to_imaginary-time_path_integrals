% \documentclass[9pt, handout]{beamer}
\documentclass[8pt,dvipsnames]{beamer}
\usetheme{Madrid}
% \usecolortheme{beaver}
\usepackage{tikz}
\usetikzlibrary{positioning,calc,arrows,decorations.pathmorphing,intersections,shapes,snakes}
\usepackage[font={small,sf},labelfont={bf},labelsep=endash]{caption}
\usepackage{sansmath}
\usepackage{xcolor}
% \usepackage[usenames, dvipsnames]{color}
\usepackage{graphicx}
\graphicspath{{figs/}}
\usepackage{framed}
\usepackage{amsmath,amssymb}
\usepackage{siunitx}
\usepackage{mathabx}
\usepackage{tasks}
\usepackage{multicol}
\usepackage{braket}
\usepackage{tikz}
\usetikzlibrary{mindmap}

\usepackage{datetime}
\yyyymmdddate
% \renewcommand{\dateseparator}{-}

\makeatletter
\def\mathcolor#1#{\@mathcolor{#1}}
\def\@mathcolor#1#2#3{%
  \protect\leavevmode
  \begingroup
    \color#1{#2}#3%
  \endgroup
}
\makeatother

\makeatletter
\let\@@magyar@captionfix
\relax\makeatother


%%%%%%%%%%%%%%%%%%%%%%%%%%%%%%%%%%%%%%%%%%%%%%%%%%%%%%%%%%%%
% Beamer Customization
%%%%%%%%%%%%%%%%%%%%%%%%%%%%%%%%%%%%%%%%%%%%%%%%%%%%%%%%%%%%
% Center the frame title
\setbeamertemplate{frametitle}[default][center]
\definecolor{MidnightBlue}{rgb}{0.2, 0.2, 0.7}
\setbeamercolor{frametitle}{fg=white,bg=MidnightBlue}
\setbeamercolor{block title}{bg=red!30,fg=black}

% Change the FONTSIZE and FONTCOLOR of frametitle
% \setbeamercolor{frametitle}{fg=blue}
\setbeamerfont{frametitle}{size=\large}
\setbeamerfont{institute}{size=\normalsize}

% Customize symbols and colors for items
% \setbeamertemplate{itemize item}{\color{yellow}$\blacksquare$}
% \setbeamercolor{itemize subitem}{fg=black}

\setbeamertemplate{itemize/enumerate body begin}{\normalsize}
\setbeamertemplate{itemize/enumerate subbody}{size=\small}
\setbeamercolor{itemize/enumerate subbody}{fg=black}

\setbeamertemplate{blocks}[default]

% Change Beamer ALERT color
% \setbeamercolor{alerted text}{fg=red}

\AtBeginSection[]
{
  \begin{frame}<beamer>{Outline}
    % \tableofcontents[currentsection,currentsubsection]
    \tableofcontents[currentsection]
  \end{frame}
}

%%%%%%%%%%%% add logo to each slide
% \logo{
%     \includegraphics[width=0.6cm]{logo.jpg}
%     \hskip5pt
%     \includegraphics[height=0.6cm]{name.jpg}
%     \hskip2in
% }

%%%%%%%%%%%% add logo to each frame title
% \addtobeamertemplate{frametitle}{}{%
% \begin{tikzpicture}[remember picture,overlay]
%     \node[anchor=north east,yshift=2pt] at (current page.south west) {
%       \includegraphics[height=1.0cm]{logo.jpg}
%       \hskip5pt
%       \includegraphics[height=1.0cm]{name.jpg}
%     };
% \end{tikzpicture}
% }
%%%%%%%%%%%%%%%%%%%%%%%%%%%%%%%%%%%%%%%%%%%%%%%%%%%%%%%%%%%%
\newcommand{\angstrom}{\text{\normalfont\AA}}
\newcommand{\abinitio}[0]{\textit{ab initio}}
\newcommand{\schro}[0]{Schr\"odinger}

\newcommand{\rt}{(\mathbf{r},t)}
\newcommand{\rR}{(\mathbf{r},\mathbf{R})}
\newcommand{\prR}{(\mathbf{r};\mathbf{R})}
\newcommand{\Rt}{(\mathbf{R},t)}
\newcommand{\rRt}{(\mathbf{r},\mathbf{R},t)}
\newcommand{\hamil}[0]{\hat{\cal H}}
\newcommand{\rvec}[0]{\mathbf{r}}
\newcommand{\svec}[0]{\mathbf{s}}
\newcommand{\Rvec}[0]{\mathbf{R}}

\newcommand{\dt}[1]{\mathrm{d}#1}
\newcommand{\onehalf}[0]{\frac{1}{2}}

\newcommand{\BOsup}[0]{\textnormal{\tiny BO}}
% \newcommand{\action}{{\cal A}}

\newcommand{\EYSR}{Elliot-Yafet}
\newcommand{\DPSR}{D'yakonov-Perel'}

\renewcommand{\Re}{\operatorname{Re}}
\renewcommand{\Im}{\operatorname{Im}}

\newcommand{\shortminus}{\text{-}}

\makeatletter
\newcommand*{\rom}[1]{\expandafter\@slowromancap\romannumeral #1@}
\makeatother
\newcommand*{\info}[4][16.3]{%
  \node[
  annotation, #3, scale=0.65, text width = #1em,
  inner sep = 6pt
  ] at (#2) {%
  \list{$\bullet$}{\topsep=0pt\itemsep=0pt\parsep=0pt
    \parskip=0pt\labelwidth=8pt\leftmargin=8pt
    \itemindent=0pt\labelsep=2pt}%
    #4
  \endlist
  };
}

\newcommand{\tikzmark}[1]{\tikz[baseline,remember picture] \coordinate (#1) {};}

\makeatletter
\renewcommand*\env@matrix[1][\arraystretch]{%
  \edef\arraystretch{#1}%
  \hskip -\arraycolsep
  \let\@ifnextchar\new@ifnextchar
  \array{*\c@MaxMatrixCols c}}
\makeatother

\newcommand*\colvec[3][]{
    \begin{pmatrix}[1.2]\ifx\relax#1\relax\else#1\\\fi#2\\#3\end{pmatrix}
}

\newenvironment {annotatedFigure}[1]{
  \centering
  \begin{tikzpicture}
    \node[anchor=south west,inner sep=0] (image) at (0,0) {#1};
        \begin{scope}[x={(image.south east)},y={(image.north west)}]}
    {\end{scope}
  \end{tikzpicture}}
%%%%%%%%%%%%%%%%%%%%%%%%%%%%%%%%%%%%%%%%%%%%%%%%%%%%%%%%%%%%
% title page
\title[]{
  Introduction to Imaginary-Time Path Integrals
}
\author[Q.J. Zheng]{
  \textcolor{red}{Qijing Zheng} \\
  % \smallskip
  % Supervisor: Prof. Jin Zhao
}
\institute[D.P. USTC]{
  % Hefei National Laboratory for Physical Sciences at the Microscale\\
  % (HFNL)\\
  Department of Physics \\
  \medskip
  University of Science and Technology of China \\
  \smallskip
  \includegraphics[width=2.0cm]{logo.jpg}
}
% \titlegraphic{
%     % \pgfuseimage{titlegraphic}
%     \includegraphics[width=2.0cm]{logo.jpg}
%     % \vskip5pt
%     % \includegraphics[height=0.8cm]{name.jpg}
% }
% \date{\today}
% \date{May 31, 2016}
\date{Jan 19, 2019}

%%%%%%%%%%%%%%%%%%%%%%%%%%%%%%%%%%%%%%%%%%%%%%%%%%%%%%%%%%%%%
\begin{document}
% Title pagee
% \maketitle and \titlepage will do the same thing, here I choose \titlepage
\begin{frame}
  \titlepage
\end{frame}

%%%%%%%%%%%%%%%%%%%%%%%%%%%%%%%%%%%%%%%%%%%%%%%%%%%%%%%%%%%%
% Outline
\begin{frame}
  \frametitle{Outline}
  \tableofcontents
\end{frame}

%%%%%%%%%%%%%%%%%%%%%%%%%%%%%%%%%%%%%%%%%%%%%%%%%%%%%%%%%%%%
% Intro
\section{Introduction}
%%%%%%%%%%%%%%%%%%%%%%%%%%%%%%%%%%%%%%%%%%%%%%%%%%%%%%%%%%%%
\begin{frame}
  \frametitle{How to model the dynamics of electrons and nuclei from \abinitio?}

  The time-dependent \schro{} equation:
  $$
  i\hbar {\partial\Psi\rRt\over\partial t} = \hamil\rR\Psi\rRt
  $$
  \medskip
  In practice, \emph{approximations} have to be made!
  \footnote{
      ``\textit{Ab initio} molecular dynamics'', Mariana Rossi, DFT Workshop 2017
  }

  \begin{center}
    \begin{tikzpicture}[scale=0.6, transform shape]
      \path [
        mindmap,
        text = white,
        level 1 concept/.append style =
          {font=\Large\bfseries},
        level 2 concept/.append style =
          {font=\small\bfseries},
        level 3 concept/.append style =
          {font=\small\bfseries},
        root/.style= {
          concept, ball color=blue,
          % font=\Huge\bfseries
        },
        BS1/.style = {concept, concept color=green!50!black},
        BS2/.style = {concept, concept color=blue!50!black},
        BS3/.style = {concept, concept color=red!90!black},
        BS4/.style = {concept, concept color=olive!90!black},
        BS5/.style = {concept, concept color=magenta!90!black},
        BS6/.style = {concept, concept color=orange!90!black},
        BS7/.style = {concept, concept color=cyan!90!black}
      ]
      node (BOA) [root] {
        \huge Born-Oppenheimer\\ Approximation?
      } 
        child[concept color=green!50!black, nodes={BS1}, grow=0] {
          node (BOYES) {\large\bfseries Quantum Nuclei?}  [clockwise from=90]
            child [concept color=pink, nodes={BS5}] {
              node (bomd) {\small\bfseries Born-Oppenheimer MD}}
            child [concept color=pink, nodes={BS5}] {
              node (cpmd) {\small Car-Parrinello MD} }
            child [concept color=orange, nodes={BS4}] {
              node (pimd) {\small Path-Integrals}}
            child [concept color=orange, nodes={BS4}] {
              node (etc)  {\small \ldots}}
          }
        child [concept color=magenta, nodes={BS3}, grow=180] {
          node (BONO) {\large\bfseries Quantum Nuclei?} [counterclockwise from=90]
            {
              child [concept color=yellow, nodes={BS6}] {
                node (EF) {\small Exact Factorizations}}
              child [concept color=yellow, nodes={BS6}] {
                node (BD) {\small Bohmian Dynamics}}
              child [concept color=blue!60!white, nodes={BS7}] {
                node (EH) {\small Ehrenfest Dynamics}}
              child [concept color=blue!60!white, nodes={BS7}] {
                node (SH) {\small Surface Hopping}}
            }
          };
          
          \path (BOA)  --  node [above=3pt] {\large YES} (BOYES);
          \path (BOA)  --  node [above=3pt] {\large NO} (BONO);

          \path (BOYES)  --  node [align=center, sloped, above=3pt] {\small\bfseries NO} (bomd);
          \path (BOYES)  --  node [align=center, sloped, above=3pt] {\small\bfseries NO} (cpmd);
          \path (BOYES)  --  node [align=center, sloped, above=3pt] {\small\bfseries YES} (pimd);
          \path (BOYES)  --  node [align=center, sloped, above=3pt] {\small\bfseries YES} (etc);

          \path (BONO)  --  node [align=center, sloped, above=3pt] {\small\bfseries YES} (EF);
          \path (BONO)  --  node [align=center, sloped, above=3pt] {\small\bfseries YES} (BD);
          \path (BONO)  --  node [align=center, sloped, above=3pt] {\small\bfseries NO} (EH);
          \path (BONO)  --  node [align=center, sloped, above=3pt] {\small\bfseries NO} (SH);
    \end{tikzpicture}
  \end{center}
   
\end{frame}
%%%%%%%%%%%%%%%%%%%%%%%%%%%%%%%%%%%%%%%%%%%%%%%%%%%%%%%%%%%%
\begin{frame}
  \frametitle{Nuclear Quantum Effects}
  
\end{frame}



%%%%%%%%%%%%%%%%%%%%%%%%%%%%%%%%%%%%%%%%%%%%%%%%%%%%%%%%%%%%
\begin{frame}[plain,c]
  \begin{center}
    \Huge Thank you!
  \end{center}
\end{frame}
%%%%%%%%%%%%%%%%%%%%%%%%%%%%%%%%%%%%%%%%%%%%%%%%%%%%%%%%%%%%%
\end{document}

%%% Local Variables:
%%% mode: latex
%%% TeX-master: t
%%% End:
